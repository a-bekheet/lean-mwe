% Blueprint content for lean-mwe
% Organized into 4 chapters mirroring the Lean module hierarchy.

\chapter{Vector Calculus Foundation}
\label{chap:vectorcalc}

This chapter establishes the vector calculus infrastructure on $\R^3$
built atop Mathlib's Fréchet derivative \texttt{fderiv}.

\section{Basic Types}

\begin{definition}[Vec3]
\label{def:Vec3}
\lean{MaxwellWave.Vec3}
\leanok
A point or vector in $\R^3$: the type $\mathrm{Fin}\,3 \to \R$.
\end{definition}

\begin{definition}[ScalarField]
\label{def:ScalarField}
\lean{MaxwellWave.ScalarField}
\leanok
\uses{def:Vec3}
A scalar field $f : \R^3 \to \R$.
\end{definition}

\begin{definition}[VectorField]
\label{def:VectorField}
\lean{MaxwellWave.VectorField}
\leanok
\uses{def:Vec3}
A vector field $\vect{F} : \R^3 \to \R^3$.
\end{definition}

\begin{definition}[TDScalarField]
\label{def:TDScalarField}
\lean{MaxwellWave.TDScalarField}
\leanok
\uses{def:Vec3}
A time-dependent scalar field $f : \R \to \R^3 \to \R$.
\end{definition}

\begin{definition}[TDVectorField]
\label{def:TDVectorField}
\lean{MaxwellWave.TDVectorField}
\leanok
\uses{def:Vec3}
A time-dependent vector field $\vect{F} : \R \to \R^3 \to \R^3$.
\end{definition}

\section{Partial Derivatives}

\begin{definition}[basisVec]
\label{def:basisVec}
\lean{MaxwellWave.basisVec}
\leanok
The $i$-th standard basis vector $\vect{e}_i = \mathrm{Pi.single}\;i\;1$.
\end{definition}

\begin{definition}[partialDeriv]
\label{def:partialDeriv}
\lean{MaxwellWave.partialDeriv}
\leanok
\uses{def:ScalarField, def:basisVec}
Partial derivative of a scalar field:
$\pd{f}{x_i}(x) = (\mathrm{fderiv}\;\R\;f\;x)(\vect{e}_i)$.
\end{definition}

\begin{definition}[partialDerivComp]
\label{def:partialDerivComp}
\lean{MaxwellWave.partialDerivComp}
\leanok
\uses{def:VectorField, def:basisVec}
Partial derivative of a vector field component:
$\pd{F_j}{x_i}(x) = (\mathrm{fderiv}\;\R\;(y \mapsto F(y)_j)\;x)(\vect{e}_i)$.
\end{definition}

\begin{definition}[partialDeriv2]
\label{def:partialDeriv2}
\lean{MaxwellWave.partialDeriv2}
\leanok
\uses{def:partialDeriv, def:basisVec}
Second partial derivative of a scalar field: $\pdd{f}{x_i}$.
\end{definition}

\begin{definition}[partialDerivComp2]
\label{def:partialDerivComp2}
\lean{MaxwellWave.partialDerivComp2}
\leanok
\uses{def:partialDerivComp, def:basisVec}
Second partial derivative of a vector field component: $\pdd{F_j}{x_i}$.
\end{definition}

\begin{definition}[timeDeriv]
\label{def:timeDeriv}
\lean{MaxwellWave.timeDeriv}
\leanok
\uses{def:TDScalarField}
Time derivative of a time-dependent scalar field:
$\pd{f}{t}(t, x) = \mathrm{deriv}\;(s \mapsto f(s, x))\;t$.
\end{definition}

\begin{definition}[timeDerivComp]
\label{def:timeDerivComp}
\lean{MaxwellWave.timeDerivComp}
\leanok
\uses{def:TDVectorField}
Time derivative of a time-dependent vector field component:
$\pd{F_j}{t}(t, x) = \mathrm{deriv}\;(s \mapsto F(s, x)_j)\;t$.
\end{definition}

\begin{definition}[timeDeriv2]
\label{def:timeDeriv2}
\lean{MaxwellWave.timeDeriv2}
\leanok
\uses{def:timeDeriv}
Second time derivative of a scalar field: $\pdd{f}{t}$.
\end{definition}

\begin{definition}[timeDerivComp2]
\label{def:timeDerivComp2}
\lean{MaxwellWave.timeDerivComp2}
\leanok
\uses{def:timeDerivComp}
Second time derivative of a vector field component: $\pdd{F_j}{t}$.
\end{definition}

\section{Vector Calculus Operators}

\begin{definition}[gradient]
\label{def:gradient}
\lean{MaxwellWave.gradient}
\leanok
\uses{def:partialDeriv}
Gradient of a scalar field:
$(\grad f)(x)_i = \pd{f}{x_i}(x)$.
\end{definition}

\begin{definition}[divergence]
\label{def:divergence}
\lean{MaxwellWave.divergence}
\leanok
\uses{def:partialDerivComp}
Divergence of a vector field:
$(\divg \vect{F})(x) = \sum_{i=0}^{2} \pd{F_i}{x_i}(x)$.
\end{definition}

\begin{definition}[curl]
\label{def:curl}
\lean{MaxwellWave.curl}
\leanok
\uses{def:partialDerivComp}
Curl of a vector field:
$(\curl \vect{F})_k = \varepsilon_{kij}\,\pd{F_j}{x_i}$,
defined by explicit pattern matching on the three components.
\end{definition}

\begin{definition}[scalarLaplacian]
\label{def:scalarLaplacian}
\lean{MaxwellWave.scalarLaplacian}
\leanok
\uses{def:partialDeriv2}
Scalar Laplacian:
$(\lapl_s f)(x) = \sum_{i=0}^{2} \pdd{f}{x_i}(x)$.
\end{definition}

\begin{definition}[vectorLaplacian]
\label{def:vectorLaplacian}
\lean{MaxwellWave.vectorLaplacian}
\leanok
\uses{def:partialDerivComp2}
Vector Laplacian:
$(\lapl \vect{F})(x)_j = \sum_{i=0}^{2} \pdd{F_j}{x_i}(x)$.
\end{definition}

\section{Smoothness Predicates}

\begin{definition}[IsC2Scalar]
\label{def:IsC2Scalar}
\lean{MaxwellWave.IsC2Scalar}
\leanok
\uses{def:ScalarField}
A scalar field is $C^2$: $\mathrm{ContDiff}\;\R\;2\;f$.
\end{definition}

\begin{definition}[IsC2Vector]
\label{def:IsC2Vector}
\lean{MaxwellWave.IsC2Vector}
\leanok
\uses{def:VectorField}
A vector field is $C^2$: each component $y \mapsto F(y)_j$ is $\mathrm{ContDiff}\;\R\;2$.
\end{definition}

\section{Fundamental Lemma and Vector Identities}

\begin{lemma}[Symmetry of mixed partials]
\label{lem:fderiv_apply_comm}
\lean{MaxwellWave.fderiv_apply_comm}
\leanok
\uses{def:IsC2Scalar}
For $C^2$ scalar fields, mixed partial derivatives commute (Clairaut/Schwarz):
\[
\frac{\partial}{\partial w}\!\left(\pd{f}{v}\right)\!(x)
= \frac{\partial}{\partial v}\!\left(\pd{f}{w}\right)\!(x).
\]
Connects to Mathlib's \texttt{IsSymmSndFDerivAt} via the chain rule for CLM evaluation.
\end{lemma}

\begin{theorem}[Curl of gradient is zero]
\label{thm:curl_gradient_eq_zero}
\lean{MaxwellWave.curl_gradient_eq_zero}
\leanok
\uses{def:curl, def:gradient, def:IsC2Scalar, lem:fderiv_apply_comm}
$\curl(\grad f) = 0$ for $C^2$ scalar fields.
Each component $\pd{}{x_i}\pd{f}{x_j} - \pd{}{x_j}\pd{f}{x_i} = 0$
by symmetry of mixed partials.
\end{theorem}

\begin{theorem}[Divergence of curl is zero]
\label{thm:divergence_curl_eq_zero}
\lean{MaxwellWave.divergence_curl_eq_zero}
\leanok
\uses{def:divergence, def:curl, def:IsC2Vector, lem:fderiv_apply_comm}
$\divg(\curl \vect{F}) = 0$ for $C^2$ vector fields.
Expands to a sum of mixed partials that cancel pairwise.
\end{theorem}

\begin{theorem}[Curl-curl identity]
\label{thm:curl_curl_eq_grad_div_sub_laplacian}
\lean{MaxwellWave.curl_curl_eq_grad_div_sub_laplacian}
\leanok
\uses{def:curl, def:gradient, def:divergence, def:vectorLaplacian, def:partialDeriv, def:IsC2Vector, lem:fderiv_apply_comm}
$\curl(\curl \vect{F}) = \grad(\divg \vect{F}) - \lapl \vect{F}$.
The critical identity for deriving wave equations. Proved component-by-component
using symmetry of mixed partials.
\end{theorem}

\section{Curl Linearity}

\begin{lemma}[IsC2Vector.differentiableAt]
\label{lem:IsC2Vector_differentiableAt}
\lean{MaxwellWave.IsC2Vector.differentiableAt}
\leanok
\uses{def:IsC2Vector}
$C^2$ vector field components are differentiable at every point.
\end{lemma}

\begin{theorem}[Curl distributes over negation]
\label{thm:curl_neg}
\lean{MaxwellWave.curl_neg}
\leanok
\uses{def:curl}
$\curl(-\vect{F}) = -(\curl \vect{F})$.
\end{theorem}

\begin{theorem}[Curl distributes over addition]
\label{thm:curl_add}
\lean{MaxwellWave.curl_add}
\leanok
\uses{def:curl}
$\curl(\vect{F} + \vect{G}) = \curl \vect{F} + \curl \vect{G}$,
given differentiability of components.
\end{theorem}

\begin{theorem}[Curl distributes over scalar multiplication]
\label{thm:curl_const_mul}
\lean{MaxwellWave.curl_const_mul}
\leanok
\uses{def:curl}
$\curl(c\,\vect{F}) = c\,(\curl \vect{F})$,
given differentiability of components.
\end{theorem}


\chapter{Electromagnetic Waves}
\label{chap:maxwell}

Maxwell's equations in linear, isotropic, homogeneous media and the derivation
of wave equations for vacuum, dielectric, and conducting media.

\section{Medium Parameters}

\begin{definition}[Medium]
\label{def:Medium}
\lean{MaxwellWave.Medium}
\leanok
Parameters of a linear, isotropic, homogeneous electromagnetic medium:
permittivity $\varepsilon > 0$, permeability $\mu > 0$, conductivity $\sigma \ge 0$.
\end{definition}

\begin{definition}[Medium.waveSpeed]
\label{def:Medium_waveSpeed}
\lean{MaxwellWave.Medium.waveSpeed}
\leanok
\uses{def:Medium}
Wave speed in the medium: $v = 1/\sqrt{\mu\varepsilon}$.
\end{definition}

\begin{definition}[Medium.waveSpeedSq]
\label{def:Medium_waveSpeedSq}
\lean{MaxwellWave.Medium.waveSpeedSq}
\leanok
\uses{def:Medium}
Squared wave speed: $v^2 = 1/(\mu\varepsilon)$.
\end{definition}

\begin{lemma}[Medium.waveSpeedSq\_pos]
\label{lem:Medium_waveSpeedSq_pos}
\lean{MaxwellWave.Medium.waveSpeedSq_pos}
\leanok
\uses{def:Medium_waveSpeedSq}
$v^2 > 0$ since $\mu, \varepsilon > 0$.
\end{lemma}

\begin{lemma}[Medium.mu\_epsilon\_pos]
\label{lem:Medium_mu_epsilon_pos}
\lean{MaxwellWave.Medium.mu_epsilon_pos}
\leanok
\uses{def:Medium}
$\mu\varepsilon > 0$.
\end{lemma}

\begin{lemma}[Medium.mu\_ne\_zero]
\label{lem:Medium_mu_ne_zero}
\lean{MaxwellWave.Medium.mu_ne_zero}
\leanok
\uses{def:Medium}
$\mu \ne 0$.
\end{lemma}

\begin{lemma}[Medium.epsilon\_ne\_zero]
\label{lem:Medium_epsilon_ne_zero}
\lean{MaxwellWave.Medium.epsilon_ne_zero}
\leanok
\uses{def:Medium}
$\varepsilon \ne 0$.
\end{lemma}

\begin{definition}[vacuum]
\label{def:vacuum}
\lean{MaxwellWave.vacuum}
\leanok
\uses{def:Medium}
Vacuum medium: $\varepsilon = \varepsilon_0$, $\mu = \mu_0$, $\sigma = 0$.
\end{definition}

\begin{definition}[dielectric]
\label{def:dielectric}
\lean{MaxwellWave.dielectric}
\leanok
\uses{def:Medium}
A lossless dielectric: general $\varepsilon, \mu$ with $\sigma = 0$.
\end{definition}

\begin{definition}[conductor]
\label{def:conductor}
\lean{MaxwellWave.conductor}
\leanok
\uses{def:Medium}
A conducting medium: $\sigma > 0$ introduces damping.
\end{definition}

\section{Maxwell's Equations}

\begin{definition}[MaxwellSystem]
\label{def:MaxwellSystem}
\lean{MaxwellWave.MaxwellSystem}
\leanok
\uses{def:Medium, def:TDVectorField, def:TDScalarField, def:divergence, def:curl, def:timeDerivComp, def:IsC2Vector}
Maxwell's equations in a linear medium, packaging: Gauss's law ($\divg \vect{E} = \rho/\varepsilon$),
no monopoles ($\divg \vect{B} = 0$), Faraday ($\curl \vect{E} = -\pd{\vect{B}}{t}$),
and Amp\`{e}re-Maxwell ($\curl \vect{B} = \mu(\vect{J}_{\mathrm{free}} + \sigma\vect{E}) + \mu\varepsilon\pd{\vect{E}}{t}$).
\end{definition}

\begin{definition}[SourceFreeMaxwell]
\label{def:SourceFreeMaxwell}
\lean{MaxwellWave.SourceFreeMaxwell}
\leanok
\uses{def:MaxwellSystem}
A source-free Maxwell system: $\rho = 0$, $\vect{J}_{\mathrm{free}} = 0$.
\end{definition}

\begin{lemma}[SourceFreeMaxwell.gauss\_simplified]
\label{lem:gauss_simplified}
\lean{MaxwellWave.SourceFreeMaxwell.gauss_simplified}
\leanok
\uses{def:SourceFreeMaxwell, def:divergence}
In source-free systems, $\divg \vect{E} = 0$.
\end{lemma}

\begin{lemma}[SourceFreeMaxwell.ampere\_simplified]
\label{lem:ampere_simplified}
\lean{MaxwellWave.SourceFreeMaxwell.ampere_simplified}
\leanok
\uses{def:SourceFreeMaxwell, def:curl, def:timeDerivComp}
In source-free systems, $\curl \vect{B} = \mu\sigma\vect{E} + \mu\varepsilon\pd{\vect{E}}{t}$.
\end{lemma}

\section{Wave Equation Derivation}

\begin{definition}[SufficientlySmooth]
\label{def:SufficientlySmooth}
\lean{MaxwellWave.SufficientlySmooth}
\leanok
\uses{def:SourceFreeMaxwell, def:IsC2Vector, def:timeDerivComp}
Smoothness hypotheses for the wave equation derivation: curl-time commutativity,
$C^2$ conditions for fields and their time derivatives, and differentiability
of time derivatives.
\end{definition}

\begin{theorem}[General wave equation for E]
\label{thm:general_wave_equation_E}
\lean{MaxwellWave.general_wave_equation_E}
\leanok
\uses{def:SourceFreeMaxwell, def:SufficientlySmooth, def:vectorLaplacian, def:timeDerivComp, def:timeDerivComp2, thm:curl_curl_eq_grad_div_sub_laplacian, lem:gauss_simplified, lem:ampere_simplified, thm:curl_neg}
\[
\lapl \vect{E} = \mu\varepsilon\,\pdd{\vect{E}}{t} + \mu\sigma\,\pd{\vect{E}}{t}.
\]
Derived by taking curl of Faraday, applying curl-curl identity with $\divg\vect{E}=0$,
and substituting Amp\`{e}re's law.
\end{theorem}

\begin{theorem}[General wave equation for B]
\label{thm:general_wave_equation_B}
\lean{MaxwellWave.general_wave_equation_B}
\leanok
\uses{def:SourceFreeMaxwell, def:SufficientlySmooth, def:vectorLaplacian, def:timeDerivComp, def:timeDerivComp2, thm:curl_curl_eq_grad_div_sub_laplacian, lem:ampere_simplified, thm:curl_neg, thm:curl_add, thm:curl_const_mul}
\[
\lapl \vect{B} = \mu\varepsilon\,\pdd{\vect{B}}{t} + \mu\sigma\,\pd{\vect{B}}{t}.
\]
Derived by taking curl of Amp\`{e}re, substituting Faraday, and using curl linearity.
\end{theorem}

\begin{theorem}[Vacuum wave equation for E]
\label{thm:vacuum_wave_equation_E}
\lean{MaxwellWave.vacuum_wave_equation_E}
\leanok
\uses{def:vacuum, thm:general_wave_equation_E}
$\lapl \vect{E} = \mu_0\varepsilon_0\,\pdd{\vect{E}}{t}$ in vacuum ($\sigma = 0$).
\end{theorem}

\begin{theorem}[Vacuum wave equation for B]
\label{thm:vacuum_wave_equation_B}
\lean{MaxwellWave.vacuum_wave_equation_B}
\leanok
\uses{def:vacuum, thm:general_wave_equation_B}
$\lapl \vect{B} = \mu_0\varepsilon_0\,\pdd{\vect{B}}{t}$ in vacuum.
\end{theorem}

\begin{theorem}[Vacuum wave speed]
\label{thm:vacuum_wave_speed}
\lean{MaxwellWave.vacuum_wave_speed}
\leanok
\uses{def:vacuum, def:Medium_waveSpeed}
$c = 1/\sqrt{\mu_0\varepsilon_0}$.
\end{theorem}

\begin{theorem}[Dielectric wave equation for E]
\label{thm:dielectric_wave_equation_E}
\lean{MaxwellWave.dielectric_wave_equation_E}
\leanok
\uses{def:dielectric, thm:general_wave_equation_E}
$\lapl \vect{E} = \mu\varepsilon\,\pdd{\vect{E}}{t}$ in a lossless dielectric.
\end{theorem}

\begin{theorem}[Dielectric wave equation for B]
\label{thm:dielectric_wave_equation_B}
\lean{MaxwellWave.dielectric_wave_equation_B}
\leanok
\uses{def:dielectric, thm:general_wave_equation_B}
$\lapl \vect{B} = \mu\varepsilon\,\pdd{\vect{B}}{t}$ in a lossless dielectric.
\end{theorem}

\begin{theorem}[Dielectric wave speed squared]
\label{thm:dielectric_wave_speed_sq}
\lean{MaxwellWave.dielectric_wave_speed_sq}
\leanok
\uses{def:dielectric, def:Medium_waveSpeedSq}
$v^2 = 1/(\mu\varepsilon)$ in a dielectric.
\end{theorem}

\begin{theorem}[Conductor wave equation for E]
\label{thm:conductor_wave_equation_E}
\lean{MaxwellWave.conductor_wave_equation_E}
\leanok
\uses{def:conductor, thm:general_wave_equation_E}
$\lapl \vect{E} = \mu\varepsilon\,\pdd{\vect{E}}{t} + \mu\sigma\,\pd{\vect{E}}{t}$
(telegraph equation).
\end{theorem}

\begin{theorem}[Conductor wave equation for B]
\label{thm:conductor_wave_equation_B}
\lean{MaxwellWave.conductor_wave_equation_B}
\leanok
\uses{def:conductor, thm:general_wave_equation_B}
$\lapl \vect{B} = \mu\varepsilon\,\pdd{\vect{B}}{t} + \mu\sigma\,\pd{\vect{B}}{t}$
(telegraph equation).
\end{theorem}

\begin{theorem}[Curl of Faraday]
\label{thm:curl_of_faraday}
\lean{MaxwellWave.curl_of_faraday}
\leanok
\uses{def:SourceFreeMaxwell, def:SufficientlySmooth, def:curl, def:timeDerivComp, thm:curl_neg}
$\curl(\curl \vect{E}) = -\pd{}{t}(\curl \vect{B})$.
\end{theorem}


\chapter{Plasma Physics}
\label{chap:plasma}

Vector algebra for plasma fields, fluid operators, the Lorentz force,
two-fluid and single-fluid MHD, resistive MHD, equilibrium theory,
and tokamak/FRC configurations.

\section{Vector Algebra}

\begin{definition}[vec3Cross]
\label{def:vec3Cross}
\lean{PlasmaEquations.vec3Cross}
\leanok
\uses{def:Vec3}
Cross product of two vectors: wraps Mathlib's \texttt{crossProduct}.
\end{definition}

\begin{definition}[vec3Dot]
\label{def:vec3Dot}
\lean{PlasmaEquations.vec3Dot}
\leanok
\uses{def:Vec3}
Dot product of two vectors: wraps Mathlib's \texttt{dotProduct}.
\end{definition}

\begin{definition}[fieldCross]
\label{def:fieldCross}
\lean{PlasmaEquations.fieldCross}
\leanok
\uses{def:VectorField, def:vec3Cross}
Pointwise cross product of two vector fields: $(\vect{F}\times\vect{G})(x) = \vect{F}(x)\times\vect{G}(x)$.
\end{definition}

\begin{definition}[fieldDot]
\label{def:fieldDot}
\lean{PlasmaEquations.fieldDot}
\leanok
\uses{def:VectorField, def:vec3Dot}
Pointwise dot product of two vector fields.
\end{definition}

\begin{definition}[scalarMul]
\label{def:scalarMul}
\lean{PlasmaEquations.scalarMul}
\leanok
\uses{def:ScalarField, def:VectorField}
Scalar--vector field multiplication: $(c\vect{F})(x) = c(x)\,\vect{F}(x)$.
\end{definition}

\begin{definition}[fieldAdd]
\label{def:fieldAdd}
\lean{PlasmaEquations.fieldAdd}
\leanok
\uses{def:VectorField}
Pointwise addition of vector fields.
\end{definition}

\begin{definition}[fieldSub]
\label{def:fieldSub}
\lean{PlasmaEquations.fieldSub}
\leanok
\uses{def:VectorField}
Pointwise subtraction of vector fields.
\end{definition}

\begin{definition}[fieldNeg]
\label{def:fieldNeg}
\lean{PlasmaEquations.fieldNeg}
\leanok
\uses{def:VectorField}
Pointwise negation of a vector field.
\end{definition}

\begin{definition}[tdFieldCross]
\label{def:tdFieldCross}
\lean{PlasmaEquations.tdFieldCross}
\leanok
\uses{def:TDVectorField, def:vec3Cross}
Time-dependent pointwise cross product.
\end{definition}

\begin{definition}[tdScalarMul]
\label{def:tdScalarMul}
\lean{PlasmaEquations.tdScalarMul}
\leanok
\uses{def:TDScalarField, def:TDVectorField}
Time-dependent scalar--vector multiplication.
\end{definition}

\begin{definition}[tdFieldAdd]
\label{def:tdFieldAdd}
\lean{PlasmaEquations.tdFieldAdd}
\leanok
\uses{def:TDVectorField}
Time-dependent pointwise addition.
\end{definition}

\begin{lemma}[cross\_anticomm]
\label{lem:cross_anticomm}
\lean{PlasmaEquations.cross_anticomm}
\leanok
\uses{def:vec3Cross, def:fieldNeg}
$\vect{a}\times\vect{b} = -(\vect{b}\times\vect{a})$.
\end{lemma}

\begin{lemma}[cross\_anticomm\_field]
\label{lem:cross_anticomm_field}
\lean{PlasmaEquations.cross_anticomm_field}
\leanok
\uses{def:fieldCross, def:fieldNeg}
Anti-commutativity at the field level.
\end{lemma}

\begin{lemma}[dot\_self\_cross\_eq\_zero]
\label{lem:dot_self_cross_eq_zero}
\lean{PlasmaEquations.dot_self_cross_eq_zero}
\leanok
\uses{def:vec3Dot, def:vec3Cross}
$\vect{a}\cdot(\vect{a}\times\vect{b}) = 0$.
\end{lemma}

\begin{lemma}[dot\_cross\_self\_eq\_zero]
\label{lem:dot_cross_self_eq_zero}
\lean{PlasmaEquations.dot_cross_self_eq_zero}
\leanok
\uses{def:vec3Dot, def:vec3Cross}
$(\vect{a}\times\vect{b})\cdot\vect{b} = 0$.
\end{lemma}

\begin{lemma}[fieldDot\_self\_cross\_eq\_zero]
\label{lem:fieldDot_self_cross_eq_zero}
\lean{PlasmaEquations.fieldDot_self_cross_eq_zero}
\leanok
\uses{def:fieldDot, def:fieldCross, lem:dot_self_cross_eq_zero}
$\vect{F}(x)\cdot(\vect{F}(x)\times\vect{G}(x)) = 0$.
\end{lemma}

\begin{lemma}[fieldDot\_cross\_self\_eq\_zero]
\label{lem:fieldDot_cross_self_eq_zero}
\lean{PlasmaEquations.fieldDot_cross_self_eq_zero}
\leanok
\uses{def:fieldDot, def:fieldCross, lem:dot_cross_self_eq_zero}
$(\vect{F}(x)\times\vect{G}(x))\cdot\vect{G}(x) = 0$.
\end{lemma}

\section{Fluid Operators}

\begin{definition}[advectiveDerivScalar]
\label{def:advectiveDerivScalar}
\lean{PlasmaEquations.advectiveDerivScalar}
\leanok
\uses{def:VectorField, def:ScalarField, def:partialDeriv}
Advective derivative of a scalar field:
$(\vect{v}\cdot\grad)f = \sum_i v_i\,\pd{f}{x_i}$.
\end{definition}

\begin{definition}[advectiveDerivVector]
\label{def:advectiveDerivVector}
\lean{PlasmaEquations.advectiveDerivVector}
\leanok
\uses{def:VectorField, def:partialDerivComp}
Advective derivative of a vector field:
$((\vect{v}\cdot\grad)\vect{F})_j = \sum_i v_i\,\pd{F_j}{x_i}$.
\end{definition}

\begin{definition}[materialDerivScalar]
\label{def:materialDerivScalar}
\lean{PlasmaEquations.materialDerivScalar}
\leanok
\uses{def:timeDeriv, def:advectiveDerivScalar}
Material derivative of a scalar field:
$\Dt{f} = \pd{f}{t} + (\vect{v}\cdot\grad)f$.
\end{definition}

\begin{definition}[materialDerivVector]
\label{def:materialDerivVector}
\lean{PlasmaEquations.materialDerivVector}
\leanok
\uses{def:timeDerivComp, def:advectiveDerivVector}
Material derivative of a vector field:
$\left(\Dt{\vect{F}}\right)_j = \pd{F_j}{t} + ((\vect{v}\cdot\grad)\vect{F})_j$.
\end{definition}

\section{Lorentz Force}

\begin{definition}[LorentzForcePerVolume]
\label{def:LorentzForcePerVolume}
\lean{PlasmaEquations.LorentzForcePerVolume}
\leanok
\uses{def:ScalarField, def:VectorField, def:fieldCross}
Lorentz force per unit volume:
$\vect{f} = \rho_c\,\vect{E} + \vect{J}\times\vect{B}$.
\end{definition}

\begin{theorem}[lorentz\_dot\_B]
\label{thm:lorentz_dot_B}
\lean{PlasmaEquations.lorentz_dot_B}
\leanok
\uses{def:LorentzForcePerVolume, def:vec3Dot, lem:dot_cross_self_eq_zero}
$\vect{f}_L\cdot\vect{B} = \rho_c(\vect{E}\cdot\vect{B})$
because $(\vect{J}\times\vect{B})\cdot\vect{B} = 0$.
\end{theorem}

\section{Two-Fluid Model}

\begin{definition}[PlasmaSpecies]
\label{def:PlasmaSpecies}
\lean{PlasmaEquations.PlasmaSpecies}
\leanok
A plasma species with charge $q$ and mass $m > 0$.
\end{definition}

\begin{definition}[TwoFluidSystem]
\label{def:TwoFluidSystem}
\lean{PlasmaEquations.TwoFluidSystem}
\leanok
\uses{def:PlasmaSpecies, def:TDScalarField, def:TDVectorField, def:timeDeriv, def:divergence, def:materialDerivVector, def:fieldCross, def:partialDeriv}
Two-fluid equations for a single species: continuity and momentum.
\end{definition}

\section{Ideal MHD}

\begin{definition}[MHDConstants]
\label{def:MHDConstants}
\lean{PlasmaEquations.MHDConstants}
\leanok
Physical constants: vacuum permeability $\mu_0 > 0$ and adiabatic index $\gamma > 1$.
\end{definition}

\begin{lemma}[MHDConstants.$\mu_0$\_ne\_zero]
\label{lem:MHDConstants_mu0_ne_zero}
\lean{PlasmaEquations.MHDConstants.μ₀_ne_zero}
\leanok
\uses{def:MHDConstants}
$\mu_0 \ne 0$.
\end{lemma}

\begin{lemma}[MHDConstants.$\mu_0$\_pos]
\label{lem:MHDConstants_mu0_pos}
\lean{PlasmaEquations.MHDConstants.μ₀_pos}
\leanok
\uses{def:MHDConstants}
$\mu_0 > 0$.
\end{lemma}

\begin{lemma}[MHDConstants.$\gamma$\_pos]
\label{lem:MHDConstants_gamma_pos}
\lean{PlasmaEquations.MHDConstants.γ_pos}
\leanok
\uses{def:MHDConstants}
$\gamma > 0$.
\end{lemma}

\begin{definition}[IdealMHD]
\label{def:IdealMHD}
\lean{PlasmaEquations.IdealMHD}
\leanok
\uses{def:MHDConstants, def:TDScalarField, def:TDVectorField, def:timeDeriv, def:timeDerivComp, def:divergence, def:curl, def:materialDerivVector, def:advectiveDerivScalar, def:fieldCross, def:vec3Cross, def:partialDeriv, def:IsC2Vector}
The ideal MHD system: mass conservation, momentum ($\rho\Dt{\vect{v}} = \vect{J}\times\vect{B} - \grad p$),
adiabatic energy, induction ($\pd{\vect{B}}{t} = \curl(\vect{v}\times\vect{B})$),
$\divg\vect{B} = 0$, and Amp\`{e}re's law ($\curl\vect{B} = \mu_0\vect{J}$).
\end{definition}

\begin{theorem}[Ideal induction from Faraday + Ohm]
\label{thm:ideal_induction_from_faraday_ohm}
\lean{PlasmaEquations.IdealMHD.ideal_induction_from_faraday_ohm}
\leanok
\uses{def:IdealMHD, def:curl, def:timeDerivComp, def:vec3Cross, thm:curl_neg}
If $\pd{\vect{B}}{t} = -\curl\vect{E}$ and $\vect{E} = -\vect{v}\times\vect{B}$ (ideal Ohm's law),
then $\pd{\vect{B}}{t} = \curl(\vect{v}\times\vect{B})$.
\end{theorem}

\begin{theorem}[J from Amp\`{e}re]
\label{thm:J_from_ampere}
\lean{PlasmaEquations.IdealMHD.J_from_ampere}
\leanok
\uses{def:IdealMHD, def:curl, lem:MHDConstants_mu0_ne_zero}
$\vect{J} = (1/\mu_0)\,\curl\vect{B}$.
\end{theorem}

\begin{theorem}[div B preserved]
\label{thm:div_B_preserved}
\lean{PlasmaEquations.IdealMHD.div_B_preserved}
\leanok
\uses{def:IdealMHD, def:divergence}
$\divg\vect{B} = 0$ at all times (from the solenoidal axiom).
\end{theorem}

\section{MHD Equilibrium}

\begin{definition}[MHDEquilibrium]
\label{def:MHDEquilibrium}
\lean{PlasmaEquations.MHDEquilibrium}
\leanok
\uses{def:MHDConstants, def:ScalarField, def:VectorField, def:partialDeriv, def:fieldCross, def:curl, def:divergence, def:IsC2Vector}
Static MHD equilibrium: $\grad p = \vect{J}\times\vect{B}$
with $\curl\vect{B} = \mu_0\vect{J}$ and $\divg\vect{B} = 0$.
\end{definition}

\begin{theorem}[B dot grad p = 0]
\label{thm:B_dot_grad_p_eq_zero}
\lean{PlasmaEquations.MHDEquilibrium.B_dot_grad_p_eq_zero}
\leanok
\uses{def:MHDEquilibrium, def:vec3Dot, def:gradient, def:fieldCross}
$\vect{B}\cdot\grad p = 0$: magnetic field lines lie on pressure surfaces.
\end{theorem}

\begin{theorem}[J dot grad p = 0]
\label{thm:J_dot_grad_p_eq_zero}
\lean{PlasmaEquations.MHDEquilibrium.J_dot_grad_p_eq_zero}
\leanok
\uses{def:MHDEquilibrium, def:vec3Dot, def:gradient, lem:fieldDot_self_cross_eq_zero}
$\vect{J}\cdot\grad p = 0$: current lines lie on pressure surfaces.
\end{theorem}

\begin{theorem}[Force balance cross form]
\label{thm:force_balance_cross_form}
\lean{PlasmaEquations.MHDEquilibrium.force_balance_cross_form}
\leanok
\uses{def:MHDEquilibrium, def:partialDeriv, def:fieldCross, def:curl, lem:MHDConstants_mu0_ne_zero}
$\grad p = (1/\mu_0)(\curl\vect{B})\times\vect{B}$.
\end{theorem}

\begin{theorem}[Magnetic pressure form]
\label{thm:magnetic_pressure_form}
\lean{PlasmaEquations.MHDEquilibrium.magnetic_pressure_form}
\leanok
\uses{def:MHDEquilibrium, def:partialDeriv, def:advectiveDerivVector, def:vec3Dot, thm:force_balance_cross_form}
$\grad p = (1/\mu_0)((\vect{B}\cdot\grad)\vect{B} - \grad(|\vect{B}|^2/2))$:
separates magnetic tension and magnetic pressure.
\end{theorem}

\section{Cylindrical Coordinates}

\begin{definition}[cylR]
\label{def:cylR}
\lean{PlasmaEquations.cylR}
\leanok
\uses{def:Vec3}
Major radius $R = x_0$ from a point $(R, \varphi, Z)$.
\end{definition}

\begin{definition}[cylZ]
\label{def:cylZ}
\lean{PlasmaEquations.cylZ}
\leanok
\uses{def:Vec3}
Vertical coordinate $Z = x_2$.
\end{definition}

\begin{definition}[IsAxisymmetric]
\label{def:IsAxisymmetric}
\lean{PlasmaEquations.IsAxisymmetric}
\leanok
\uses{def:ScalarField, def:partialDeriv}
A scalar field is axisymmetric: $\pd{f}{\varphi} = 0$.
\end{definition}

\begin{definition}[PoloidalFlux]
\label{def:PoloidalFlux}
\lean{PlasmaEquations.PoloidalFlux}
\leanok
\uses{def:ScalarField, def:IsC2Scalar, def:IsAxisymmetric}
Poloidal flux function $\psi(R, Z)$ with $C^2$ smoothness and axisymmetry.
\end{definition}

\begin{definition}[GradShafranovOp]
\label{def:GradShafranovOp}
\lean{PlasmaEquations.GradShafranovOp}
\leanok
\uses{def:ScalarField, def:partialDeriv, def:partialDeriv2, def:cylR}
The Grad-Shafranov operator:
$\Delta^*\psi = \pdd{\psi}{R} - \frac{1}{R}\pd{\psi}{R} + \pdd{\psi}{Z}$.
\end{definition}

\section{Grad-Shafranov Equation}

\begin{definition}[GradShafranovEquation]
\label{def:GradShafranovEquation}
\lean{PlasmaEquations.GradShafranovEquation}
\leanok
\uses{def:MHDConstants, def:PoloidalFlux, def:GradShafranovOp, def:cylR}
The Grad-Shafranov equation for tokamak equilibrium:
$\Delta^*\psi = -\mu_0 R^2 p'(\psi) - g(\psi)g'(\psi)$.
\end{definition}

\begin{definition}[GradShafranovEquation.pressure]
\label{def:GS_pressure}
\lean{PlasmaEquations.GradShafranovEquation.pressure}
\leanok
\uses{def:GradShafranovEquation}
Pressure profile $p(x) = p(\psi(x))$.
\end{definition}

\begin{theorem}[Pressure on flux surfaces]
\label{thm:gs_pressure_flux_surface}
\lean{PlasmaEquations.GradShafranovEquation.gs_pressure_flux_surface}
\leanok
\uses{def:GS_pressure}
If $\psi(x) = \psi(y)$, then $p(x) = p(y)$: pressure is constant on flux surfaces.
\end{theorem}

\begin{theorem}[g on flux surfaces]
\label{thm:gs_g_flux_surface}
\lean{PlasmaEquations.GradShafranovEquation.gs_g_flux_surface}
\leanok
\uses{def:GradShafranovEquation}
$g$ is constant on flux surfaces.
\end{theorem}

\begin{definition}[GradShafranovEquation.magField]
\label{def:GS_magField}
\lean{PlasmaEquations.GradShafranovEquation.magField}
\leanok
\uses{def:GradShafranovEquation, def:cylR, def:partialDeriv}
Magnetic field from flux:
$\vect{B} = (-\frac{1}{R}\pd{\psi}{Z},\; \frac{g(\psi)}{R},\; \frac{1}{R}\pd{\psi}{R})$.
\end{definition}

\begin{theorem}[GS implies equilibrium]
\label{thm:gs_is_equilibrium}
\lean{PlasmaEquations.GradShafranovEquation.gs_is_equilibrium}
\leanok
\uses{def:GradShafranovEquation, def:GS_pressure, def:GS_magField, def:partialDeriv, def:fieldCross}
The Grad-Shafranov equation implies $\grad p = \vect{J}\times\vect{B}$
in cylindrical geometry ($R \ne 0$).
\end{theorem}

\begin{theorem}[Solov'ev linearization]
\label{thm:gs_solovev_linear}
\lean{PlasmaEquations.GradShafranovEquation.gs_solovev_linear}
\leanok
\uses{def:GradShafranovEquation, def:GradShafranovOp, def:cylR}
When $p(\psi) = p_0\psi$ and $g^2(\psi) = g_0^2 + 2\alpha\psi$ are linear,
the GS equation becomes $\Delta^*\psi = -\mu_0 R^2 p_0 - \alpha$.
\end{theorem}

\section{Resistive MHD}

\begin{definition}[OhmsLaw]
\label{def:OhmsLaw}
\lean{PlasmaEquations.OhmsLaw}
\leanok
Ohm's law parameters: resistivity $\eta \ge 0$.
\end{definition}

\begin{definition}[ResistiveMHD]
\label{def:ResistiveMHD}
\lean{PlasmaEquations.ResistiveMHD}
\leanok
\uses{def:MHDConstants, def:OhmsLaw, def:TDScalarField, def:TDVectorField, def:timeDeriv, def:timeDerivComp, def:divergence, def:curl, def:materialDerivVector, def:advectiveDerivScalar, def:fieldCross, def:vec3Cross, def:partialDeriv, def:IsC2Vector}
Resistive MHD: same as ideal but with $\vect{E} + \vect{v}\times\vect{B} = \eta\vect{J}$.
Induction becomes $\pd{\vect{B}}{t} = \curl(\vect{v}\times\vect{B} - \eta\vect{J})$.
\end{definition}

\begin{theorem}[Resistive reduces to ideal]
\label{thm:resistive_reduces_to_ideal}
\lean{PlasmaEquations.ResistiveMHD.resistive_reduces_to_ideal}
\leanok
\uses{def:ResistiveMHD, def:OhmsLaw, def:curl, def:timeDerivComp, def:vec3Cross}
When $\eta = 0$, the resistive induction equation reduces to the ideal one.
\end{theorem}

\begin{theorem}[Resistive induction diffusion form]
\label{thm:resistive_induction_diffusion_form}
\lean{PlasmaEquations.ResistiveMHD.resistive_induction_diffusion_form}
\leanok
\uses{def:ResistiveMHD, def:curl, def:timeDerivComp, def:vec3Cross, thm:curl_add, thm:curl_const_mul}
$\pd{\vect{B}}{t} = \curl(\vect{v}\times\vect{B}) - \eta\,\curl\vect{J}$:
splits convection and diffusion terms.
\end{theorem}

\section{FRC Equilibrium}

\begin{definition}[FRCEquilibrium]
\label{def:FRCEquilibrium}
\lean{PlasmaEquations.FRCEquilibrium}
\leanok
\uses{def:MHDConstants}
FRC equilibrium: radial profiles $p(R)$, $B_z(R)$, $B_\theta(R)$
satisfying radial pressure balance with field reversal at the separatrix.
\end{definition}

\begin{theorem}[FRC no-toroidal balance]
\label{thm:frc_no_toroidal_balance}
\lean{PlasmaEquations.FRCEquilibrium.frc_no_toroidal_balance}
\leanok
\uses{def:FRCEquilibrium}
With $B_\theta = 0$: $\frac{dp}{dR} + \frac{1}{\mu_0}B_z\frac{dB_z}{dR} = 0$.
\end{theorem}

\begin{theorem}[FRC total pressure conservation]
\label{thm:frc_total_pressure_conservation}
\lean{PlasmaEquations.FRCEquilibrium.frc_total_pressure_conservation}
\leanok
\uses{def:FRCEquilibrium}
$p(R) = (B_{\mathrm{ext}}^2 - B_z(R)^2)/(2\mu_0)$
when total pressure is conserved.
\end{theorem}

\begin{theorem}[FRC beta at separatrix]
\label{thm:frc_beta_at_separatrix}
\lean{PlasmaEquations.FRCEquilibrium.frc_beta_at_separatrix}
\leanok
\uses{def:FRCEquilibrium}
$\beta = 1$ at the separatrix: $2\mu_0 p(R_s) = B_{\mathrm{ext}}^2$.
\end{theorem}

\begin{definition}[RotamakDrive]
\label{def:RotamakDrive}
\lean{PlasmaEquations.RotamakDrive}
\leanok
RMF drive parameters: amplitude $B_\omega > 0$, frequency $\omega > 0$.
\end{definition}

\begin{definition}[RotamakSystem]
\label{def:RotamakSystem}
\lean{PlasmaEquations.RotamakSystem}
\leanok
\uses{def:FRCEquilibrium, def:RotamakDrive, def:MHDConstants}
A Rotamak: FRC sustained by a rotating magnetic field with 1D Amp\`{e}re's law
$dB_z/dR = -\mu_0 J_\theta$.
\end{definition}

\begin{theorem}[Rotamak Amp\`{e}re consistency]
\label{thm:rotamak_ampere_consistency}
\lean{PlasmaEquations.RotamakSystem.rotamak_ampere_consistency}
\leanok
\uses{def:RotamakSystem, lem:MHDConstants_mu0_pos}
Inside the separatrix where $B_z$ decreases ($dB_z/dR < 0$),
the toroidal current $J_\theta > 0$.
\end{theorem}


\chapter{Incompressible Navier-Stokes}
\label{chap:ns}

The incompressible Navier-Stokes equations, Euler equations (inviscid limit),
vorticity dynamics, and the pressure Poisson equation.

\section{Equations of Motion}

\begin{definition}[FluidConstants]
\label{def:FluidConstants}
\lean{NavierStokes.FluidConstants}
\leanok
Fluid constants: density $\rho > 0$, dynamic viscosity $\mu \ge 0$.
\end{definition}

\begin{lemma}[FluidConstants.$\rho$\_ne\_zero]
\label{lem:FluidConstants_rho_ne_zero}
\lean{NavierStokes.FluidConstants.ρ_ne_zero}
\leanok
\uses{def:FluidConstants}
$\rho \ne 0$.
\end{lemma}

\begin{lemma}[FluidConstants.$\rho$\_pos]
\label{lem:FluidConstants_rho_pos}
\lean{NavierStokes.FluidConstants.ρ_pos}
\leanok
\uses{def:FluidConstants}
$\rho > 0$.
\end{lemma}

\begin{definition}[FluidConstants.$\nu$]
\label{def:FluidConstants_nu}
\lean{NavierStokes.FluidConstants.ν}
\leanok
\uses{def:FluidConstants}
Kinematic viscosity $\nu = \mu/\rho$.
\end{definition}

\begin{lemma}[FluidConstants.$\nu$\_nonneg]
\label{lem:FluidConstants_nu_nonneg}
\lean{NavierStokes.FluidConstants.ν_nonneg}
\leanok
\uses{def:FluidConstants_nu}
$\nu \ge 0$.
\end{lemma}

\begin{definition}[IncompressibleNS]
\label{def:IncompressibleNS}
\lean{NavierStokes.IncompressibleNS}
\leanok
\uses{def:FluidConstants, def:TDVectorField, def:TDScalarField, def:divergence, def:materialDerivVector, def:partialDeriv, def:vectorLaplacian, def:IsC2Vector, def:IsC2Scalar}
Incompressible Navier-Stokes: $\divg\vect{v} = 0$ and
$\rho\Dt{\vect{v}} = -\grad p + \mu\lapl\vect{v} + \vect{f}$.
\end{definition}

\begin{theorem}[Momentum per unit mass]
\label{thm:momentum_per_unit_mass}
\lean{NavierStokes.IncompressibleNS.momentum_per_unit_mass}
\leanok
\uses{def:IncompressibleNS, def:FluidConstants_nu, def:materialDerivVector, def:partialDeriv, def:vectorLaplacian, lem:FluidConstants_rho_ne_zero}
$\Dt{\vect{v}} = -\frac{1}{\rho}\grad p + \nu\lapl\vect{v} + \frac{1}{\rho}\vect{f}$.
\end{theorem}

\begin{theorem}[Continuity from incompressible]
\label{thm:continuity_from_incompressible}
\lean{NavierStokes.IncompressibleNS.continuity_from_incompressible}
\leanok
\uses{def:IncompressibleNS, def:divergence}
$\divg(\rho\vect{v}) = 0$ since $\rho$ is constant and $\divg\vect{v} = 0$.
\end{theorem}

\section{Euler Equations}

\begin{theorem}[Euler momentum]
\label{thm:euler_momentum}
\lean{NavierStokes.IncompressibleNS.euler_momentum}
\leanok
\uses{def:IncompressibleNS, def:materialDerivVector, def:partialDeriv}
When $\mu = 0$: $\rho\Dt{\vect{v}} = -\grad p + \vect{f}$.
\end{theorem}

\begin{theorem}[Euler momentum per unit mass]
\label{thm:euler_momentum_per_unit_mass}
\lean{NavierStokes.IncompressibleNS.euler_momentum_per_unit_mass}
\leanok
\uses{thm:momentum_per_unit_mass}
When $\mu = 0$: $\Dt{\vect{v}} = -\frac{1}{\rho}\grad p + \frac{1}{\rho}\vect{f}$.
\end{theorem}

\section{Vorticity}

\begin{definition}[vorticity]
\label{def:vorticity}
\lean{NavierStokes.vorticity}
\leanok
\uses{def:TDVectorField, def:curl}
Vorticity: $\vect{\omega}(t,x) = \curl\vect{v}(t,x)$.
\end{definition}

\begin{theorem}[div vorticity = 0]
\label{thm:div_vorticity_eq_zero}
\lean{NavierStokes.div_vorticity_eq_zero}
\leanok
\uses{def:vorticity, def:divergence, def:IncompressibleNS, thm:divergence_curl_eq_zero}
$\divg\vect{\omega} = 0$: vorticity is solenoidal.
\end{theorem}

\begin{theorem}[curl pressure gradient = 0]
\label{thm:curl_pressure_gradient_eq_zero}
\lean{NavierStokes.curl_pressure_gradient_eq_zero}
\leanok
\uses{def:IncompressibleNS, def:curl, def:gradient, thm:curl_gradient_eq_zero}
$\curl(\grad p) = 0$.
\end{theorem}

\begin{definition}[NSSmoothness]
\label{def:NSSmoothness}
\lean{NavierStokes.NSSmoothness}
\leanok
\uses{def:IncompressibleNS, def:curl, def:timeDerivComp, def:basisVec}
Smoothness bundle for vorticity/pressure derivations:
$C^3$ velocity, curl-time commutativity, and divergence-time commutativity.
\end{definition}

\begin{lemma}[partialDerivComp\_differentiable]
\label{lem:partialDerivComp_differentiable}
\lean{NavierStokes.partialDerivComp_differentiable}
\leanok
\uses{def:IsC2Vector, def:partialDerivComp}
Partial derivatives of $C^2$ vector fields are differentiable.
\end{lemma}

\begin{theorem}[curl of advective term]
\label{thm:curl_advective_eq}
\lean{NavierStokes.curl_advective_eq}
\leanok
\uses{def:curl, def:advectiveDerivVector, def:IsC2Vector, def:divergence, lem:fderiv_apply_comm, lem:partialDerivComp_differentiable}
$\curl((\vect{v}\cdot\grad)\vect{v}) = (\vect{v}\cdot\grad)\vect{\omega} - (\vect{\omega}\cdot\grad)\vect{v}$
for divergence-free $\vect{v}$. The cornerstone identity for vorticity dynamics.
\end{theorem}

\begin{theorem}[curl commutes with Laplacian]
\label{thm:curl_vectorLaplacian_eq}
\lean{NavierStokes.curl_vectorLaplacian_eq}
\leanok
\uses{def:IncompressibleNS, def:NSSmoothness, def:curl, def:vectorLaplacian, lem:fderiv_apply_comm, lem:partialDerivComp_differentiable}
$\curl(\lapl\vect{v}) = \lapl(\curl\vect{v})$
for sufficiently smooth incompressible velocity fields.
\end{theorem}

\begin{theorem}[Vorticity transport equation]
\label{thm:vorticity_transport}
\lean{NavierStokes.vorticity_transport}
\leanok
\uses{def:IncompressibleNS, def:NSSmoothness, def:curl, def:advectiveDerivVector, def:vectorLaplacian, def:timeDerivComp, def:FluidConstants_nu, thm:momentum_per_unit_mass, thm:curl_advective_eq, thm:curl_vectorLaplacian_eq, thm:curl_gradient_eq_zero, thm:curl_neg, thm:curl_add, thm:curl_const_mul}
\[
\pd{\vect{\omega}}{t} + (\vect{v}\cdot\grad)\vect{\omega}
= (\vect{\omega}\cdot\grad)\vect{v} + \nu\lapl\vect{\omega} + \frac{1}{\rho}\curl\vect{f}.
\]
The vortex stretching term $(\vect{\omega}\cdot\grad)\vect{v}$ is unique to 3D.
\end{theorem}

\section{Pressure Poisson Equation}

\begin{theorem}[Pressure Poisson equation]
\label{thm:pressure_poisson}
\lean{NavierStokes.pressure_poisson}
\leanok
\uses{def:IncompressibleNS, def:NSSmoothness, def:scalarLaplacian, def:partialDerivComp, def:divergence, lem:fderiv_apply_comm, lem:partialDerivComp_differentiable}
\[
\lapl_s p = -\rho\sum_{i,k}\pd{v_k}{x_i}\pd{v_i}{x_k} + \divg\vect{f}.
\]
The pressure is determined as a constraint enforcing incompressibility.
\end{theorem}
